\chapter{Kurzfassung}


\begin{german}

Diese Masterarbeit implementiert einen Compiler für MiniC\verb++| unter der Verwendung von Java-Technologien zur Erzeugung von Java-Bytecode. Für das Frontend wird der kombinierte Lexer- und Parsergenerator ANTLR verwendet, während das Backend die ObjectWeb ASM Bibliothek für die Bytecodegenerierung nutzt. Ähnlich zu Generatoren wie Coco-2 generiert ANTLR den Quellcode für einen Parser auf der Grundlage einer vorgegebenen Grammatik. ANTLR kann Code in mehreren Hostsprachen generieren; für diese Arbeit wurde Java gewählt. 

ANTLR ist ein Open-Source-Tool für die Spracherkennung. Es wurde erstmals 1992 veröffentlicht und wird seitdem von Terrence Parr kontinuierlich weiterentwickelt. In der aktuellen Version von ANTLR wird der adaptive-LL(*)oder ALL(*)-Parsing-Algorithmus verwendet. Dieser Algorithmus führt die Grammatikanalyse zur Parse-Zeit durch und überkommt dadurch die Einschränkungen früherer Versionen. ALL(*) erfordert nicht die Angabe einer maximalen Anzahl von Lookahead-Tokens. Stattdessen passt es sich an den aktuellen Kontext an und adaptiert die Anzahl der Lookahead-Token entsprechend.

Um einen abstrakten Syntaxbaum (AST) mit ANTLR zu erzeugen, werden drei Methoden unterstützt: Das Visitor-Pattern, das Listener-Pattern und die Verwendung einer attributierten Grammatik (ATG). Das Compiler-Frontend wird mit allen drei Methoden implementiert. Die drei Implementierungen werden in mehreren Aspekten verglichen und auf dieser Grundlage wird eine Empfehlung ausgesprochen.

ObjectWeb ASM ist eine in Java geschriebene Bibliothek zur Erzeugung von Java-Bytecode. Sie ist Open-Source und wird seit 2002 von Eric Bruneton entwickelt. Die Bibliothek wird in Compilern verwendet, die Java-Bytecode ausgeben, wie z.B. dem Kotlin-Compiler. Zur Generierung von Bytecode wird die auf dem Visitor-Pattern basierende API verwendet. 

Alle drei Frontend-Implementierungen sind mit dem Backend zu einer Anwendung verbunden, die Java-Bytecode aus MiniC\verb|++|-Code erzeugen kann. 

\end{german}
