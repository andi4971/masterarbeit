\chapter{Kurzfassung}

Peer-to-Peer-Netzwerke bieten eine Alternative zum klassischen Client-Server-Modell, um Daten auszutauschen. In Peer-to-Peer-Netzwerken kommunizieren alle Clients miteinander. Dadurch kann auf den Server als zentrale Schnittstelle verzichtet werden. Diese Charakteristik ermöglicht es Peer-to-Peer-Netzwerken zu funktionieren, obwohl einzelne Teilnehmer im Netzwerk ausfallen. Zudem nutzen Peer-to-Peer-Netzwerke die (meist bei traditionellem Filesharing ungenutzte) Upload-Bandbreite der einzelnen Clients. 

Diese Bachelorarbeit setzt sich detailliert mit Peer-to-Peer-Netzwerken auseinander. Dabei werden zuerst bekannte Peer-to-Peer-Netzwerke vorgestellt und deren Charakteristiken erläutert. Weiters wird gezeigt, wo Unternehmen und Organisationen Peer-to-Peer-Netzwerke einsetzen. Unterschieden wird hierbei zwischen frei verfügbaren Netzwerken und von Unternehmen eigens entwickelten Netzwerken. Abschließend wird ein Client für das Netzwerk BitTorrent entwickelt. Dieser Client ist in der Lage, unter Verwendung des BitTorrent-Protokolls eine Datei von anderen Peers herunter- und hochzuladen. Dadurch wird gezeigt wie der Datenaustausch in einem Peer-to-Peer-Netzwerk auf technischer Ebene funktioniert und welche Technologien dazu benötigt werden.