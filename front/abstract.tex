\chapter{Abstract}


\begin{english} %switch to English language rules
Peer-to-peer networks offer an alternative to the classic client-server model for exchanging data. In peer-to-peer networks, all clients communicate with each other. This means that the server, as the central element, can be omitted. This characteristic enables peer-to-peer networks to function even if individual participants in the network fail. In addition, peer-to-peer networks use the upload bandwidth of the individual clients, which is usually unused in traditional file sharing. 

This bachelor thesis deals in detail with peer-to-peer networks. First, known peer-to-peer networks are introduced and their characteristics are explained. Then it is shown where companies and organisations utilize peer-to-peer networks. A distinction is made between freely available networks and networks developed by companies themselves. Finally, a client for the BitTorrent network is developed. This client is able to exchange a file with other peers using the BitTorrent protocol. This shows how data exchange in a peer-to-peer network works on a technical level and which technologies are required for this.
\end{english}

