\chapter{Introduction}
\label{cha:introduction}

\section{Motivation}

Compilers function as the backbone for computer programming. A compiler takes care of translating human-readable source code into something a computer can understand. This allows the application developer to focus only on writing the application, without having to worry about the technicalities of the concrete computer where the software will run on. For one programming language there may exist multiple compilers targeting different kinds of computers. This allows the same source code to run for example on Linux and Windows. This flexibility saves the developer a lot of work, because they don't need to rewrite their application in the case they also want to target another operating system. Furthermore, there also exist compilers that target virtual machines like the Java Virtual Machine (JVM). Generating code for a virtual machine has the advantage that there don't need to be compilers written for every target operating system. Instead, for each operating system an implementation of the virtual machine is provided.

The process of compiling source code begins in the frontend of the compiler. In this step the source code is read, and an abstract syntax tree (AST) is constructed. The AST is a runtime representation of the source code in memory. It contains only the necessary information that is later on needed to generate machine code. The process of constructing the AST is based the grammar of the programming language. Based on this grammar a lexer and parser are either written manually or get generated by a parser generator tool like ANTLR. In the case of ANTLR the generated parser and lexer construct a full parse tree from the input. From this an AST can be constructed using for example the visitor-pattern. 

The AST functions then as the input for the backend of the compiler. In this section the machine code for the target system is generated. In the case of the JVM this is the so called Bytecode. The Bytecode could be written by hand, however this is rather difficult. For this a detailed understanding of the instruction set is needed, and the actual generation would have to be performed on a byte array. Therefore, APIs exist, that provide an abstraction layer to the code generation. One API for Bytecode generation is the open source project ObjectWeb ASM or just ASM. It provides an API that utilizes the visitor-pattern to generate Bytecode instructions. 
