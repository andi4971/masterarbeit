\chapter{Einleitung}
\label{cha:Einleitung}

\section{Motivation}

In den letzten 25 Jahren entstand eine Vielzahl an Peer-to-Peer Filesharing-Netz\-werken. Immer schnellere Internetgeschwindigkeiten und eine steigende Anzahl an Haushalten mit Internetzugang führten zu einem vermehrten Interesse, über das Internet Daten auszutauschen. Vor allem das Netzwerk Napster sorgte für öffentliche Aufmerksamkeit. Das Netzwerk zum Teilen von Musik erreichte innerhalb kurzer Zeit eine Nutzerzahl von 60 Millionen \parencite{poblocki2001napster}. Aufgrund von Copyright-Verletzungen wurde der Betrieb von Napster schließlich gerichtlich unterbunden\footnote[1]{Naspter ist jetzt ein kommerzieller Musikstreamingdienst.}. Jedoch enstanden zur etwa gleichen Zeit mehrere andere Netzwerke, die bis heute bestehen. Zu diesen Netzwerken zählen unter anderem BitTorrent, Gnutella und Soulseek. 

Mittlerweile hat sich BitTorrent als am weitesten verbreitetes Netzwerk durchgesetzt, wie Statistiken von \textcite{BitTorrentTrafficSandvine} zeigen. 
Mehrere Linux-Distributionen, wie unter anderem Ubuntu oder Debian, bieten den Download ihrer Installationsdateien auch mittels BitTorrent an. Für die Distributions-Hersteller und die Bezieher hat die Verwendung von BitTorrent mehrere Vorteile. Diese werden im Rahmen dieser Arbeit vorgestellt.


\section{Problemstellung}

Kosten für das Versenden von Daten über das Internet werden immer relevanter für Unternehmen. Es müssen Wege gefunden werden, große Mengen an Daten schnell und effizient an die Kunden zu senden. Eine Möglichkeit, dies zu bewerkstelligen, bilden Peer-to-Peer Filesharing-Netzwerke. Als Unternehmen ist es wichtig zu wissen, welche Vor- und Nachteile der Einsatz dieser Technologie mit sich bringt und wie andere Unternehmen damit umgehen.

\section{Zielsetzung}

Im Zuge dieser Bachelorarbeit sollen die Vor- und Nachteile von Peer-to-Peer Filesharing-Netzwerken dargestellt werden. Dafür sollen anhand von verschiedenen Aspekten die Unterschiede zu traditionellem Filesharing belichtet werden. Weiters soll ein Überblick geschaffen werden, in welchen Bereichen aktuell der kommerzielle Einsatz von Peer-to-Peer Netzwerken stattfindet.
Zudem soll ein Client für das meist verbreitete Peer-to-Peer Netzwerk BitTorrent implementiert werden. Der Fokus liegt hier darauf zu zeigen wie das dem Netzwerk zugrunde liegende Protokoll aufgebaut ist und der Datenaustausch zwischen den Peers funktioniert. Es ist nicht Ziel, einen Client zu entwickeln, welcher die neuesten Erweiterungen des Protokolls implementiert oder die höchstmöglichen Download-Raten erreicht.