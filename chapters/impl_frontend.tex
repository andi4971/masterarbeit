\chapter{Implementation Frontend}

This chapter explains the implementation of the frontend of the compiler. First the additional technologies that are used in the development of the frontend are listed. Then the AST and symboltable are explained. In the following section the ANTLR grammar is shown. Based on this grammar three implementations for the AST transformation are explained: Visitor-pattern, listener-pattern and via an attributed grammar. 

\section{Used Technologies}

In this section the additional technologies used for the development of the frontend are explained. This includes chosen the programming language and the additional libraries used.

\subsection{Kotlin}

For the implementation the programming language Kotlin is chosen. Kotlin can be compiled to bytecode, which makes it possible to use Java libraries in Kotlin code. Kotlin has advantages over Java in some aspects. For example, null safety\footnote{https://kotlinlang.org/docs/null-safety.html} is implemented into the language via explicit nullability within it's type system. This requires the caller of a field to explicitly handle nullable fields and thus reduces the risk of a null reference exception. 

\subsection{AspectJ}

The compiler frontend utilizes AspectJ in it's handling of semantic errors. AspectJ is a library that enables aspect oriented programming in Java. Aspect oriented programming makes it possible to handle cross-cutting concerns in a central place without having to modify code in other areas. It can be used for compile time and runtime weaving of cross-cutting concerns. In the compiler frontend, runtime weaving using annotations is used.  

\subsection{ANTLR Preview Plugin}

During development with the JetBrains IntelliJ IDE, the ANTLR preview plugin\footnote{https://plugins.jetbrains.com/plugin/7358-antlr-v4} is used. The plugin developed by the ANTLR creator Terrance Parr, offers various features that enhance the process of creating and working with ANTLR. When developing an ANTLR grammar, syntax highlighting and checking for syntactic and semantic errors is provided. The included navigation window inside the IDE further enables testing of the grammar, without having to generate the combined lexer and parser manually first. To generate the combined lexer and parser the plugin includes a tool window which includes common configuration settings.

\section{ANTLR Grammar}

The ANTLR grammar is based on the MiniC++ grammar in EBNF-form. This grammar is transformed into the ANTLR grammar syntax. ANTLR grammars are stored inside \texttt{.g4} files. Each rule inside the grammar is delimited by a semicolon. 

\subsection{Header Section}

At the top of the grammar file, the name of the grammar is specified. In this case \texttt{minicpp}. In this section options can also be specified, like the implementation language of the lexer and parser or the package/namespace of the generated code. These and other options can also be specified via command line options during the generation. In this case the necessary options are specified in the tool window of the ANTLR preview plugin. 

%if in need add picture of tool window

\subsection{Terminal Classes and Comments}

The grammar contains three terminal classes shown in listing \ref{lst:ANTLRMiniCppTermClass}. The \texttt{IDENT} terminal class is used for all identifiers and requires them to start with a letter followed by an arbitrary number of letters and digits. For integer number the \texttt{INT} terminal class specifies a sequence of one or more digits. Signs are handled in the parser rules. Strings are defined as a sequence of characters starting and ending with double quotes. All characters except the special characters for line end are allowed. The comment and whitespace handling is performed by the \texttt{WS}, \verb|LINE_COMMENT| and \verb|BLOCK_COMMENT| lexical rules. These are special rules that when matched tell the parser to skip them and therefore not include them in the syntax tree. 

\begin{AntlrCode}[float,numbers=none,caption=Terminal classes of the MiniC++ ANTLR grammar., label=lst:ANTLRMiniCppTermClass]
IDENT:          [a-zA-Z_][a-zA-Z_0-9]*;
INT:            [0-9]+;
STRING:         '"' (~[\r\n"] | '""')* '"';
WS:             [ \t\n\r]+ -> skip;
LINE_COMMENT:   '//' ~[\r\n]* -> skip;
BLOCK_COMMENT:  '/*' .*? '*/' -> skip;
\end{AntlrCode}


\subsection{Root}

The top rules of the grammar are shown in listing \ref{lst:ANTLRMiniCppTop}. The root rule \texttt{miniCpp} contains zero or more elements of the rule \texttt{miniCppEntry} followed by the lexical rule \texttt{EOF}. \texttt{EOF} is a default lexical rule provided by ANTLR signaling the end of the file. \texttt{miniCppEntry} defines the elements that can be used at the top level of the miniC\verb|++| source file. These are variable and constant definitions and function declarations and definitions. \texttt{SEM} is the lexical rule defining a semicolon. 

\begin{AntlrCode}[float,numbers=none,caption=Top rules of the MiniC++ ANTLR grammar., label=lst:ANTLRMiniCppTop]
miniCpp:     (miniCppEntry)* EOF;
miniCppEntry:     constDef
                | varDef
                | funcDecl
                | funcDef
                | SEM
                ;
\end{AntlrCode}



\subsection{Variables and Constants}

Listing \ref{lst:ANTLRMiniCppVarConstDef} shows the parser rules variable and constant definitions. Both definitions can have multiple entries, which are separated by a comma. In the case of a constant definition entry, the initialization value is required. For a variable this is optional. The \texttt{STAR} optional lexical rule classifies a field as an array if present. The \texttt{initOption} parser rule consists of three production alternatives specifying the possible initialization values. 

\begin{AntlrCode}[float,numbers=none,caption=Variable and constant defintions of the MiniC++ ANTLR grammar., label=lst:ANTLRMiniCppVarConstDef]
constDef:        CONST type constDefEntry (COMMA constDefEntry)* SEM ;
constDefEntry:   IDENT init ;

varDef:          type varDefEntry (COMMA varDefEntry)* SEM ;
varDefEntry:     STAR? IDENT (init)? ;

init:            EQUAL  initOption ;
initOption:      BOOLEAN      #BooleanInit
               | NULLPTR      #NullptrInit
               | (SIGN)? INT  #IntInit
               ;
\end{AntlrCode}

\subsection{Function Declaration and Definition}

The rules for a function declaration and definition are shown in listing \ref{lst:ANTLRMiniCppFuncDeclDef}. In MiniC\verb|++| to call a function there must be at least a declaration of the function further ahead in the source code. Function declarations and definitions both start with the function head that consists of the return type, identifier and an optional parameter list. In the case of a function definition, the function head is followed the \texttt{block} rule, which contains the method's body. 

The parameter list can consist either of a single entry, the \texttt{void} type, or of one or more actual input parameters. Each parameter specified by the \texttt{formParListEntry} rule, consists of a type, an optional star and brackets indicating an array followed by the identifier of the parameter.


\begin{AntlrCode}[float,numbers=none,caption=Function declaration and defintion of the MiniC++ ANTLR grammar., label=lst:ANTLRMiniCppFuncDeclDef]
funcDecl:         funcHead SEM;
funcDef:          funcHead block;
funcHead:         type STAR? IDENT LPAREN formParList? RPAREN;
formParList:      (     VOID
                  |     formParListEntry (COMMA formParListEntry)*
                  );
formParListEntry: type STAR? IDENT (BRACKETS)?;
\end{AntlrCode}

\subsection{Statements}

The parser rule defining a statement is shown in listing \ref{lst:ANTLRMiniCppStatAlt}. The statement rules serves as a container for all concrete statement types. For example, the \texttt{ifStat} rule can be seen in listing \ref{lst:ANTLRMiniCppStatIf}. It consists of the \texttt{if} keyword followed by the condition in parentheses and a statement which should be executed if the condition is met. Optionally an else statement can be specified. This rule does not suffer from the \textit{dangling else} problem. In such cases ANTLR resolves the ambiguity by always choosing the first successful production. 

\begin{AntlrCode}[float,numbers=none,caption=Statement rule and it's production alternatives of the MiniC++ ANTLR grammar., label=lst:ANTLRMiniCppStatAlt]
stat:  ( emptyStat  | breakStat
       | blockStat  | exprStat
       | ifStat     | whileStat
       | inputStat  | outputStat
       | deleteStat | returnStat
       );
\end{AntlrCode}

\begin{AntlrCode}[float,numbers=none,caption=If Statement rule  of the MiniC++ ANTLR grammar., label=lst:ANTLRMiniCppStatIf]
ifStat:      'if' LPAREN expr RPAREN stat elseStat?;
elseStat:    'else' stat;
\end{AntlrCode}



\subsection{Expressions}

Part of the grammar for expressions is shown in listing \ref{lst:ANTLRMiniCppExprBool}. At the top of every expression is an \texttt{orExpr} followed by zero or more \texttt{exprEntry} elements. In case an \texttt{exprEntry} is present, the expression performs one or mor assignments. Each \texttt{exprEntry} consists of an assignment operator signalizing the type of assignment, and an \texttt{orExpr} that provides the value to be assigned. The \verb|orExpr| and \verb|andExpr| rules implement their respective boolean operators. The \verb|relExpr| rule consists of a \verb|simpleExpr| and zero or more \verb|relExprEntry| elements. The \verb|relExprEntry| rule handles relative expressions with the \verb|relExprOperator| rule, which contains the relative operators like greater or less than. The \verb|simpleExpr| rule begins with an optional sign that is relevant for integer values, followed by a term and zero or more \verb|simpleExprEntry| elements. The \verb|simpleExprEntry| rule consists of a sign followed by a term. The precedence rules for arithmetic operations are realized inside the grammar.



\begin{AntlrCode}[float,numbers=none,caption=Expression rules for assignment and boolean operations of the MiniC++ ANTLR grammar., label=lst:ANTLRMiniCppExprBool]
expr:                 orExpr (exprEntry)*;
exprEntry:            exprAssign orExpr;
exprAssign:           EQUAL      #EqualAssign
                    | ADD_ASSIGN #AddAssign
                    | SUB_ASSIGN #SubAssign
                    | MUL_ASSIGN #MulAssign
                    | DIV_ASSIGN #DivAssign
                    | MOD_ASSIGN #ModAssign
                    ;
orExpr:             andExpr ( '||' andExpr )*;
andExpr:            relExpr ( '&&' relExpr )*;
relExpr:            simpleExpr
                    ( relExprEntry )*;
relExprEntry:       relOperator simpleExpr;
simpleExpr:         (SIGN)?
                    term ( simpleExprEntry )*;
simpleExprEntry:    SIGN term;
\end{AntlrCode}

The rules for terms and factors are shown in listing \ref{lst:ANTLRMiniCppExprTermFact}. A term consists of a fact, which contains an optional negation making it a \verb|notFact|, followed by zero or more \verb|termEntry| elements. The \verb|termEntry| rule realizes multiplication, division and modulo operations via the \verb|termOperator| rule. The \verb|fact| rule contains six possible productions. Three types of value literals can be used as a factor: integer, boolean or null-pointer. Another option is the array initialization, defined by the \verb|#NewArrayFact| alternative. The \verb|#ExprFact| alternative allows for precedence using an expression contained in parentheses. To read the value of a variable or array, or call a function the \verb|#CallFact| alternative using the \verb|callFactEntry| rule is used. The \verb|callFactEntry| rule contains an optional increment/decrement at the beginning and end of the rule. Each \verb|INC_DEC| element has a named alias so that in the syntax tree it can be easily checked which element is null. Via the \verb|IDENT| terminal class the name of the variable or array to read can be specified. If \verb|callFactEntryOperation| is not null, then either a function call or array access is performed. Depending on the type of function that is called, parameters may be necessary. The \verb|#ActParListFactOperation| alternative allows for parameters via the optional \verb|actParList| rule. This rule consists of one or more expressions, that make up the parameters.     

\begin{AntlrCode}[float,numbers=none,caption=Expression rules for terms and factors of the MiniC++ ANTLR grammar., label=lst:ANTLRMiniCppExprTermFact]
term:             notFact (termEntry)*;
termEntry:        termOperator notFact;
termOperator:     STAR #StarOperator
                | DIV #DivOperator
                | MOD #ModOperator;
notFact:          NOT? fact;
fact:             BOOLEAN #BooleanFact
                | NULLPTR #NullptrFact
                | INT     #IntFact
                | callFactEntry         #CallFact
                | NEW type LBRACK expr RBRACK #NewArrayFact
                | LPAREN expr RPAREN          #ExprFact
                ;
callFactEntry:    preIncDec=INC_DEC?
                  IDENT
                  callFactEntryOperation?
                  postIncDec=INC_DEC?
                  ;
callFactEntryOperation:
      ( LBRACK expr    RBRACK)          #ExprFactOperation
    | ( LPAREN (actParList)?    RPAREN) #ActParListFactOperation
    ;
actParList:       expr (COMMA expr)*;
\end{AntlrCode}

\section{Abstract Syntax Tree (AST)}

